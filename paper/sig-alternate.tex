% This is "sig-alternate.tex" V2.0 May 2012
% This file should be compiled with V2.5 of "sig-alternate.cls" May 2012
%
% This example file demonstrates the use of the 'sig-alternate.cls'
% V2.5 LaTeX2e document class file. It is for those submitting
% articles to ACM Conference Proceedings WHO DO NOT WISH TO
% STRICTLY ADHERE TO THE SIGS (PUBS-BOARD-ENDORSED) STYLE.
% The 'sig-alternate.cls' file will produce a similar-looking,
% albeit, 'tighter' paper resulting in, invariably, fewer pages.
%
% ----------------------------------------------------------------------------------------------------------------
% This .tex file (and associated .cls V2.5) produces:
%       1) The Permission Statement
%       2) The Conference (location) Info information
%       3) The Copyright Line with ACM data
%       4) NO page numbers
%
% as against the acm_proc_article-sp.cls file which
% DOES NOT produce 1) thru' 3) above.
%
% Using 'sig-alternate.cls' you have control, however, from within
% the source .tex file, over both the CopyrightYear
% (defaulted to 200X) and the ACM Copyright Data
% (defaulted to X-XXXXX-XX-X/XX/XX).
% e.g.
% \CopyrightYear{2007} will cause 2007 to appear in the copyright line.
% \crdata{0-12345-67-8/90/12} will cause 0-12345-67-8/90/12 to appear in the copyright line.
%
% ---------------------------------------------------------------------------------------------------------------
% This .tex source is an example which *does* use
% the .bib file (from which the .bbl file % is produced).
% REMEMBER HOWEVER: After having produced the .bbl file,
% and prior to final submission, you *NEED* to 'insert'
% your .bbl file into your source .tex file so as to provide
% ONE 'self-contained' source file.
%
% ================= IF YOU HAVE QUESTIONS =======================
% Questions regarding the SIGS styles, SIGS policies and
% procedures, Conferences etc. should be sent to
% Adrienne Griscti (griscti@acm.org)
%
% Technical questions _only_ to
% Gerald Murray (murray@hq.acm.org)
% ===============================================================
%
% For tracking purposes - this is V2.0 - May 2012

\documentclass{sig-alternate}

\begin{document}
%
% --- Author Metadata here ---
\conferenceinfo{WOODSTOCK}{'97 El Paso, Texas USA}
%\CopyrightYear{2007} % Allows default copyright year (20XX) to be over-ridden - IF NEED BE.
%\crdata{0-12345-67-8/90/01}  % Allows default copyright data (0-89791-88-6/97/05) to be over-ridden - IF NEED BE.
% --- End of Author Metadata ---

\title{Alternate {\ttlit ACM} SIG Proceedings Paper in LaTeX
Format\titlenote{(Produces the permission block, and
copyright information). For use with
SIG-ALTERNATE.CLS. Supported by ACM.}}
\subtitle{[Extended Abstract]
\titlenote{A full version of this paper is available as
\textit{Author's Guide to Preparing ACM SIG Proceedings Using
\LaTeX$2_\epsilon$\ and BibTeX} at
\texttt{www.acm.org/eaddress.htm}}}
%
% You need the command \numberofauthors to handle the 'placement
% and alignment' of the authors beneath the title.
%
% For aesthetic reasons, we recommend 'three authors at a time'
% i.e. three 'name/affiliation blocks' be placed beneath the title.
%
% NOTE: You are NOT restricted in how many 'rows' of
% "name/affiliations" may appear. We just ask that you restrict
% the number of 'columns' to three.
%
% Because of the available 'opening page real-estate'
% we ask you to refrain from putting more than six authors
% (two rows with three columns) beneath the article title.
% More than six makes the first-page appear very cluttered indeed.
%
% Use the \alignauthor commands to handle the names
% and affiliations for an 'aesthetic maximum' of six authors.
% Add names, affiliations, addresses for
% the seventh etc. author(s) as the argument for the
% \additionalauthors command.
% These 'additional authors' will be output/set for you
% without further effort on your part as the last section in
% the body of your article BEFORE References or any Appendices.

\numberofauthors{3} %  in this sample file, there are a *total*
% of EIGHT authors. SIX appear on the 'first-page' (for formatting
% reasons) and the remaining two appear in the \additionalauthors section.
%
\author{
% You can go ahead and credit any number of authors here,
% e.g. one 'row of three' or two rows (consisting of one row of three
% and a second row of one, two or three).
%
% The command \alignauthor (no curly braces needed) should
% precede each author name, affiliation/snail-mail address and
% e-mail address. Additionally, tag each line of
% affiliation/address with \affaddr, and tag the
% e-mail address with \email.
%
% 1st. author
\alignauthor
Johannes Spie{\ss}berger\\
% 2nd. author
\alignauthor
Ralph Hoch\\
% 3rd. author
\alignauthor 
Carola Gabriel\\
}
% There's nothing stopping you putting the seventh, eighth, etc.
% author on the opening page (as the 'third row') but we ask,
% for aesthetic reasons that you place these 'additional authors'
% in the \additional authors block, viz.
% Just remember to make sure that the TOTAL number of authors
% is the number that will appear on the first page PLUS the
% number that will appear in the \additionalauthors section.

\maketitle
\begin{abstract}
Targeted advertising is one of the key revenue sources 
for internet services. While traditional approaches tried
to suggest ads to users based on statistics derived from historical data,
modern approaches try to make use of big data.
This paper tries to give a short overview of current scientific trends 
in NoSQL and big data management. As graph databases are especially fitting for 
modelling social interactions, this paper puts an emphasis on this type of NoSql.
\end{abstract}

\terms{Big Data, targeted advertising, NoSQL, graph database}

\keywords{Big Data, targeted advertising, NoSQL, graph database}

\section{Introduction}
The recent years have shown a massive growth in data 
used for analysis in a broad range of fields.
Big data as a concept for handling this amount of information
has become a huge field for scientific research and 
business models alike. One of the cornerstones of the technical
implementation of such systems is the usage of NoSQL databases.
While these are available in many different flavours, graph based approaches
are the one that differ the most from traditional database systems.
Modelling social interactions as graphs and extracting various information
is, among other applications, one of todays key techniques for targeted advertising. TODO cite http://ieeexplore.ieee.org/stamp/stamp.jsp?tp=&arnumber=1579567
While the usage of such graph databases makes for a natural abstraction
of some real life observations, the technical implementations
of the graph database itself becomes more challenging than those of
traditional RDBMs system.
The following sections will summarize some recent research concerning 
the inner workings of graph databases.

\section{GPU based frequent graph mining}
With the availability of CUDA and opencl GPUs have become
a source of cheap computation power for significantlly less money 
than general purpose CPUs. While not applicable to all types
of computational loads, there are various fields which benefit 
immensly from a high parallelization degree TODO source.
Especially loads without the need of communication between the 
threads are well suited for GPU architectures TODO source.
The following sections will describe some applications in graph databases.

Source: http://jmlr.org/proceedings/papers/v36/kessl14.pdf
Frequent graph mining describes the search of reoccuring sub patterns 
in a graph. Among various applications this technique can be used to
find similar communities within social networks TODO source.
TODO bild subgraph
As this problem has shown to be NP-complete TODO source it is especially important to
find scalable algorithms to deal with this problem.
The GPU-based graph mining algorithm combines concepts from gSpan TODO source
and DMTL TODO source. In order to achieve good performance it is necessary 
to change the data structure of the graph in main memory. Hashmaps or linked lists
are not well suited for GPUs as data is not cached and therefore access locality becomes 
a performance consideration. 
For various benchmarks the usage of GPUs has lead to a significant speedup.
TODO Benchmark bild
This work can also be extended to different graph algorithms like closed graph,
maximal graph and temporal graph mining.

\section{Differential Queries}
http://ceur-ws.org/Vol-1133/paper-34.pdf
http://download.springer.com/static/pdf/593/chp%253A10.1007%252F978-3-319-10933-6_9.pdf?auth66=1422138244_54c98542c3346ab3b48752871f074418&ext=.pdf
Abandoning the rigid predefined schema of traditional database systems is one of the 
key features of NoSQL. While offering more flexibility in terms of development,
users have a hard time gaining deep knowledge of the data and its structure.
This mostly manifests in queries giving unexpected or empty result sets.
A query in a graph database can be understood as a pattern that has to match parts of the graph.
To help the user understand why he does not get the expected result subgraphs between the
query and the dataset are calculated, so it can be shown which part of the query leads to an empty result set.
Using algorithms to detect maximum common connected subgraphs, nodes in the query can be 
partitioned into discovered and undiscovered conditions.
Due to the nature of these operations are large number of intermediate results 
may lead to performance problems.
This problem may be mitigated by introducing Top-K differential queries. 
Such queries are still modeled as graphs but have additional properties allowing for ranking
the intermediate results by relevance. These weights are given by user and can be either maximized or minimized.

\section{SLQ}
%https://www.cs.ucsb.edu/~xyan/papers/sigmod14_slq_demo.pdf
Another approach of easing the writing of graph database queries for non
professional users is SLQ language. SLQ interprets queries based on 
keywords which are then further mapped to onthologies. 
While a strict replacing of words would lead to high number of queries,
the mapping of keywords is based on their semantical closenes.
To get the most relevant results the generated queries and their transformations
are weighted. By using used feedback and query logs SLQ is able to
learn how to apply a reasonable weighting.
To create a SLQ query the user specifies it by drawing a graph
and attaching the keywords.
TODO BILD

\section{Parallel Adjacency Lists}
http://arxiv.org/pdf/1403.0701.pdf
Modelling social networks like Twitter has some particular challenges
for graph DB which come from the uneven distribution of edges between nodes.
Such graphs have the tendency to have a small number of nodes with a high number of edges
and a high number of nodes with a small number of edges. TODO zitat aus Paper 11
This makes partitioning a graph to allow for localized acces hard.
Partitioned Adjacecny Lists PAL tries to reduce the number of random accesses while minimizing the storage
space required. Edges are stored as pairs of vertices and are partitioned by a range of destination
vertices and are ordered by the source ID. These partitiones must not be of equal length but should 
be chosen such that a partition can fit into memory. This theoretically limits the number of
incoming edges a vertex may have.
As the graph connectivity is fully represented 
by the edge partitions, vertex data can be stored seperately and is partitioned 
based on the edge partitions. Retrieving verted data can be achieved by simly calculating an offset from
the edge partitions.
Edge partitions are immutable data structures and do not allow direct insertion of edges. 
To counter this problem new edges a stored in a buffer and inserted if their number exceeds a certain threshold.
During search operations these buffers are also considered.
Todo zahlen und benchmark bilder

\section{Distributed Graph Database}
%http://delivery.acm.org/10.1145/2690000/2689677/p1-dayarathna.pdf?ip=128.131.237.61&id=2689677&acc=ACTIVE%20SERVICE&key=9074CF143665B1C6.97709C79A94C9E0F.4D4702B0C3E38B35.4D4702B0C3E38B35&CFID=621086552&CFTOKEN=17755639&__acm__=1422145282_da483414101f63310da71f6d768c3e23
Current graph databases are mainly inteded to run on a single instance and therefore
face performance and availability problems. TODO Zitat aus paper 
Acadia is a distributed graph database intented to run in hybrid cloud settings.
By utilizing graph partitioning and a Master-Worker pattern it is well fit to 
operate in an environment with less than 100 workers.
Each distributed process has an associated local data store with an assigned storage quota.
When adding a new graph the partitioning happens via Hadoop and Metis and the subgraphs are then inserted into local
Neo4j instances.

\section{Conclusions}
TODO
%\end{document}  % This is where a 'short' article might terminate


%
% The following two commands are all you need in the
% initial runs of your .tex file to
% produce the bibliography for the citations in your paper.
\bibliographystyle{abbrv}
\bibliography{sigproc}  % sigproc.bib is the name of the Bibliography in this case
% You must have a proper ".bib" file
%  and remember to run:
% latex bibtex latex latex
% to resolve all references
%
% ACM needs 'a single self-contained file'!
%
%\balancecolumns % GM June 2007
% That's all folks!
\end{document}
